\documentclass{beamer}

% Top-aligning columns within a top-aligned frame
% https://tex.stackexchange.com/questions/16447/beamer-top-aligning-columns-within-a-top-aligned-frame
\makeatletter
\newenvironment{myitemize}{%
   \setlength{\topsep}{0pt}
   \setlength{\partopsep}{0pt}
   \renewcommand*{\@listi}{\leftmargin\leftmargini \parsep\z@ \topsep\z@ \itemsep\z@}
   \let\@listI\@listi
   \itemize
}{\enditemize}
\makeatother  

\usetheme{Warsaw}
\usepackage[USenglish]{babel}
\usepackage[utf8]{inputenc}
\usepackage{amssymb, amsmath}
\usepackage{bm}
\usepackage{color}
\usepackage{url}
\hypersetup{
    colorlinks,
    citecolor=blue,
    filecolor=blue,
    linkcolor=blue,
    urlcolor=blue
}

\begin{document}
\title{Born Again Neural Networks}  
\author{Radek Bartyzal}
\date{Date TBA} 
\institute{Let's talk ML in Prague}

\frame{\titlepage} 

\begin{frame}{Prior work}


\begin{block}{Ensembles}
Diverse models with similar validation performances can be often be combined to achieve predictive
power superior to each of the constituent models. \cite{cit:ensembles}
\end{block}

\begin{block}{Born again trees}
Learn a single tree that is able to recover the performance of a multiple-tree predictor. \cite{cit:bat}
\end{block}

\begin{block}{Knowledge distillation = model compression}
Transfer knowledge acquired by a learned
teacher model to a new simpler student model. \cite{cit:distill}
\end{block}



\end{frame}
%--------- END Frame 1 -------------

\begin{frame}[t]{Knolewdge distillation}

\begin{columns}[t]
\begin{column}{0.5\textwidth}
Teacher
\begin{itemize}
\item high-capacity model
\item good performance
\end{itemize}
\end{column}

\begin{column}{0.5\textwidth}
Student
\begin{itemize}
\item more compact model
\item not as good performance as the teacher but better than if it was trained without it
\end{itemize}
\end{column}

\end{columns}

\vfill
By transferring knowledge, one hopes to benefit from the student’s
compactness while suffering only minimal degradation in performance.

\end{frame}
%--------- END Frame 2 -------------

\begin{frame}{Born Again Networks}

\begin{itemize}
\item not compressing models
\item students are parameterized identically to their parents.
\item 
\end{itemize}

  Surprisingly, these born again networks (BANs), tend to outperform their
teacher models. Our experiments with born again dense networks demonstrate
state-of-the-art performance on the CIFAR-100 dataset reaching a validation error
of 15.5\% with a single model and 14.9\% with our best ensemble. Additionally, we
investigate knowledge transfer to architectures that are different, but with capacity
comparable to their teachers. In these experiments, we show that similar advantages
can be achieved by transferring knowledge between dense networks and residual
networks of similar capacity.

\end{frame}


\begin{frame}{Sources}

\begin{thebibliography}{0}

  \bibitem[1]{cit:ban} Tommaso Furlanello et al. "Born Again Neural Networks." Workshop on Meta-Learning (MetaLearn 2017) at NIPS. Accessible from: \url{http://metalearning.ml/papers/metalearn17_furlanello.pdf}
  
  \bibitem[2]{cit:stat} Breiman, Leo. "Statistical modeling: The two cultures (with comments and a rejoinder by the author)." Statistical science 16.3 (2001): 199-231. Accessible from: \url{https://projecteuclid.org/download/pdf_1/euclid.ss/1009213726\%20}
  
  \bibitem[3]{cit:ensembles} Hansen, Lars Kai, and Peter Salamon. "Neural network ensembles." IEEE transactions on pattern analysis and machine intelligence 12.10 (1990): 993-1001.
  
  \bibitem[4]{cit:bat} Breiman, Leo, and Nong Shang. "Born again trees." Accessible from: \url{https://www.stat.berkeley.edu/~breiman/BAtrees.pdf} ps (1996).
  
  \bibitem[5]{cit:distill} Hinton, Geoffrey, Oriol Vinyals, and Jeff Dean. "Distilling the knowledge in a neural network." arXiv preprint arXiv:1503.02531 (2015).
  
  
\end{thebibliography}

\end{frame}
 
 
 
\end{document}
